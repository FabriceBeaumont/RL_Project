%%%%%%%%%%%%%%%%%%%%%%%%%%%%%%%%%%%%%%%%%%%%%%%%%%%%%%%%%%%%%%%%%%%%%%%%%%%%%%%
%% HEADER
%%%%%%%%%%%%%%%%%%%%%%%%%%%%%%%%%%%%%%%%%%%%%%%%%%%%%%%%%%%%%%%%%%%%%%%%%%%%%%%
\documentclass[fontsize=12pt, paper=a4, headinclude, twoside=false, parskip=half+, pagesize=auto, numbers=noenddot, open=right, toc=listof, toc=bibliography]{scrreprt}
%\documentclass[runningheads]{report}
% PDF-Kompression
\pdfminorversion=5
\pdfobjcompresslevel=1
\usepackage[automark]{scrpage2} % Kopf- und Fußzeilen
\usepackage{setspace} % Zeilenabstand
\onehalfspacing % 1,5 Zeilen
\usepackage{scrhack} % Vermeidung einer Warnung

%% Language / Text %%%%%%%%%%%%%%%%%%%%%%%%%%%%%%%%%%%%%%%%%%%%%%%%%%%%%%%%%%%%
\usepackage[ngerman]{babel}
\usepackage[T1]{fontenc}
\usepackage[ansinew]{inputenc}
\usepackage{lmodern} % Type1-font for non-english texts and characters
\usepackage[T1]{fontenc} % Ligaturen, richtige Umlaute im PDF
\usepackage{ulem}
\usepackage{cancel}
\usepackage{adjustbox} % To roate a table
\usepackage[labelsep=endash]{caption}
%\addto\captionsenglish{\renewcommand{\figurename}{Figure}}
%\renewcommand{\figurename}{Figure}
\usepackage[figurename=Figure]{caption}
%\usepackage[utf8]{inputenc}% UTF8-Kodierung für Umlaute usw
%Darstellung deutscher Sonderzeichen (Umlaute etc)
% Schriften-Größen
%\setkomafont{chapter}{\Huge\rmfamily} % Überschrift der Ebene
%\setkomafont{section}{\Large\rmfamily}
%\setkomafont{subsection}{\large\rmfamily}
%\setkomafont{subsubsection}{\large\rmfamily}
%\setkomafont{chapterentry}{\large\rmfamily} % Überschrift der Ebene in Inhaltsverzeichnis
%\setkomafont{descriptionlabel}{\bfseries\rmfamily} % für description-Umgebungen
%\setkomafont{captionlabel}{\small\bfseries}
%\setkomafont{caption}{\small}
\usepackage{chngcntr} % for \counterwithout{subsection}{section}
\setlength\parindent{0pt}

%% Math Symbols %%%%%%%%%%%%%%%%%%%%%%%%%%%%%%%%%%%%%%%%%%%%%%%%%%%%%%%%%%%%%%%
\usepackage{amsthm}
\usepackage{amsfonts}
\usepackage{amsmath,marvosym, amssymb, mathtools} % Mathesachen
\usepackage{siunitx}%Schöne Darstellung großer Zahlen \num{}
\usepackage{mathpazo} % Palatino für Mathemodus

%% Other Packages %%%%%%%%%%%%%%%%%%%%%%%%%%%%%%%%%%%%%%%%%%%%%%%%%%%%%%%%%%%%%
\usepackage{color}                          % colors
\usepackage{framed}                         % frames
\usepackage{mdframed}                       % better frames
\usepackage{float}

%\usepackage{fancyhdr}                   % Fancy headings
%\usepackage{mathpazo,tgpagella} % auch sehr schöne Schriften

% PDF
%\usepackage[ngerman,pdfauthor={Martin Bretschneider},  pdfauthor={Martin Bretschneider}, pdftitle={LaTeX-Vorlage für Bachelorarbeiten, Masterarbeiten und Dissertationen}, breaklinks=true,baseurl={http://www.bretschneidernet.de/tips/thesislatex.html}]{hyperref}
\usepackage[final]{microtype} % mikrotypographische Optimierungen
\usepackage{url} % ermögliche Links (URLs)
\usepackage{pdflscape} % einzelne Seiten drehen können
%\usepackage[svgnames, table, hyperref]{xcolor}
%\usepackage{color}
\usepackage{ bbold } % for mathbb{1}
%\usepackage[margin=2cm]{geometry}
\usepackage{amstext}
\usepackage{bigints}
%\usepackage{amscd}
%\usepackage{amsthm}
%\usepackage{amsbsy}
%\usepackage{tikz-cd}
%\usepackage{bm}
\usepackage{cleveref}
\usepackage[most]{tcolorbox}
\usepackage{empheq}
% Tabellen
\usepackage{multirow} % Tabellen-Zellen über mehrere Zeilen
\usepackage{multicol} % mehrere Spalten auf eine Seite
\usepackage{tabularx, booktabs} % für Tabellen mit vorgegeben Größen
\usepackage{longtable} % Tabellen über mehrere Seiten
\usepackage{array}
\usepackage{makecell} % for newline in cells \makecell{text\\text}

%\renewcommand\theadalign{bc}
%\renewcommand{\cellalign/theadalign}{vl}
%\renewcommand\theadfont{\bfseries}
%\renewcommand\theadgape{\Gape[4pt]}
%\renewcommand\cellgape{\Gape[4pt]}

%  Bibliographie
\usepackage{bibgerm} % Umlaute in BibTeX
% Bilder
\usepackage{graphicx} % Bilder
\usepackage{color} % Farben
\usepackage{caption}
\usepackage{subcaption}  % mehrere Abbildungen nebeneinander/übereinander
%\usepackage[all]{hypcap} % Beim Klicken auf Links zum Bild und nicht zu Caption gehen
% Bildunterschrift
\setcapindent{0em} % kein Einrücken der Caption von Figures und Tabellen
\setcapwidth{0.9\textwidth} % Breite der Caption nur 90% der Textbreite, damit sie sich vom restlichen Text abhebt
\setlength{\abovecaptionskip}{0.2cm} % Abstand der zwischen Bild- und Bildunterschrift
% F�r Pseudocode-Darstellungen
\usepackage[ruled]{algorithm}
\usepackage[noend]{algpseudocode} 
\makeatletter
\def\BState{\State\hskip-\ALG@thistlm}
\algdef{SE}[DOWHILE]{Do}{doWhile}{\algorithmicdo}[1]{\algorithmicwhile\ #1}
\makeatother
% Quellcode
% f�r Formatierung in Quelltexten, hier im Anhang
\usepackage{listings}
\usepackage{csquotes} % für Anführungszeichen/ Zitate
\usepackage{enumerate}
\usepackage{enumitem}
%\usepackage[shortlabels]{enumerate}
\definecolor{grau}{gray}{0.25}
\lstset{
	extendedchars=true,
	basicstyle=\tiny\ttfamily,
	%basicstyle=\footnotesize\ttfamily,
	tabsize=2,
	keywordstyle=\textbf,
	%commentstyle=\color{grau},
	stringstyle=\textit,
	numbers=left,
	numberstyle=\tiny,
	% für schönen Zeilenumbruch
	breakautoindent  = true,
	breakindent      = 2em,
	breaklines       = true,
	postbreak        = ,
	prebreak         = \raisebox{-.8ex}[0ex][0ex]{\Righttorque},
}

\definecolor{javared}{rgb}{0.6,0,0} % for strings
\definecolor{javagreen}{rgb}{0.25,0.5,0.35} % for strings
\definecolor{javapurple}{rgb}{0.5,0,0.35} % for strings
\definecolor{javadocblue}{rgb}{0.25,0.35,0.75} % for strings
\lstset{
	language=Java,
	basicstyle=\fontsize{8}{7}\ttfamily,
	keywordstyle=\color{javapurple}\bfseries,
	stringstyle=\color{javared},
	commentstyle=\color{javagreen},
	morecomment=[s][\color{javadocblue}]{/**}{*/}
	numbers=left,
	numberstyle=\tiny\color{black},
	stepnumber=2,
	numbersep=10pt,
	tabsize=4,
	showspaces=false,
	showstringspaces=false
%	{Ö}{{\"O}},	{Ä}{{\"A}},	{Ü}{{\"U}},	{ß}{{\ss}},	{ö}{{\"o}},	{ä}{{\"a}},	{ü}{{\"u}},
}

%\usepackage{exsheets} %http://www.ctan.org/pkg/exsheets 

\newcommand{\IP}{\mathbb{P}} 
\newcommand{\IN}{\mathbb{N}}
\newcommand{\IR}{\mathbb{R}}
\newcommand{\IZ}{\mathbb{Z}}
\newcommand{\IQ}{\mathbb{Q}}
\newcommand{\ID}{\mathbb{D}}
\newcommand{\IC}{\mathbb{C}}
\newcommand{\IE}{\mathbb{E}}
\newcommand{\vF}{\mathcal{F}}
\newcommand{\vA}{\mathcal{A}}
\newcommand{\vD}{\mathcal{D}}
\newcommand{\vB}{\mathcal{B}}
\newcommand{\vP}{\mathcal{P}}
\newcommand{\vJ}{\mathcal{J}}
\newcommand{\vN}{\mathcal{N}}
\newcommand{\vU}{\mathcal{U}}
\newcommand{\Bin}{\mathop{\mathrm{Bin}}}
\newcommand{\Poi}{\mathop{\mathrm{Poi}}}
\newcommand{\Cov}{\mathop{\mathrm{Cov}}}
\newcommand{\Var}{\mathop{\mathrm{Var}}}
% linksbündige Fußboten
\deffootnote{1.5em}{1em}{\makebox[1.5em][l]{\thefootnotemark}}

\typearea{14} % typearea berechnet einen sinnvollen Satzspiegel (das heißt die Seitenränder usw.) siehe auch http://www.ctan.org/pkg/typearea. Diese Berechnung befindet sich am Schluss, damit die Einstellungen von oben berücksichtigt werden

\newcommand{\norm}[1]{\left\lVert#1\right\rVert}
\newcommand{\var}[1]{{\ttfamily#1}}% variable

% Eigene Befehle %%%%%%%%%%%%%%%%%%%%%%%%%%%%%%%%%%%%%%%%%%%%%%%%%5
% Matrix
\newcommand{\mat}[1]{
      {\textbf{#1}}
}
\newcommand{\todo}[1]{
      {\colorbox{red}{ TODO: #1 }}
}
\newcommand{\todotext}[1]{
      {\color{red} TODO: #1} \normalfont
}
\newcommand{\texttodo}[1]{
	{\color{red} TODO: #1} \normalfont
}
%\newcommand{\info}[1]{
%      {\colorbox{blue}{ (INFO: #1)}}
%}
% Hinweis auf Programme in Datei
\newcommand{\datei}[1]{
      {\ttfamily{#1}}
}
\newcommand{\code}[1]{
      {\ttfamily{#1}}
}
% bild mit defnierter Breite einfügen
\newcommand{\bild}[4]{
  \begin{figure}[!hbt]
    \centering
      \vspace{1ex}
      \includegraphics[width=#2]{images/#1}
      \caption[#4]{\label{img.#1} #3}
    \vspace{1ex}
  \end{figure}
}
% bild mit eigener Breite
\newcommand{\bilda}[3]{
  \begin{figure}[!hbt]
    \centering
      \vspace{1ex}
      \includegraphics{images/#1}
      \caption[#3]{\label{img.#1} #2}
      \vspace{1ex}
  \end{figure}
}

% Bild todo
\newcommand{\bildt}[2]{
  \begin{figure}[!hbt]
    \begin{center}
      \vspace{2ex}
	      \includegraphics[width=6cm]{images/todobild}
      %\caption{\label{#1} \color{red}{ TODO: #2}}
      \caption{\label{#1} \todotext{#2}}
      %{\caption{\label{#1} {\todo{#2}}}}
      \vspace{2ex}
    \end{center}
  \end{figure}
}

%\theoremstyle{definition}
%\newtheorem{aufgabe}{Aufgabe}[section]

%\theoremstyle{remark}
%\newtheorem{hinweis}[aufgabe]{Hinweis}

\newtcbtheorem[auto counter, number within = section]{myBox}{}{
	breakable, %for usage on multiple pages
	enhanced,	
	attach boxed title to top left={
		xshift=-1mm,
		yshift=-5mm,
		yshifttext=-1mm
	},
	top=1.5em,
	colback=white,
	colframe=black!75!black,
	fonttitle=\bfseries,
	boxed title style={
		sharp corners,
		size=small,
		colback=black!75!black,
		colframe=black!75!black,
	} 
}{myBox}


%\newenvironment{Definition}[2]{ \begin{myBox}[adjusted title=DEF #1: #2, sharp corners]}{\end{myBox}}
\newenvironment{Definition}[1]{ \begin{myBox}[adjusted title=DEF: #1, sharp corners]}{\end{myBox}}
%\refstepcounter{section}\par\medskip\noindent
%\newenvironment{Theorem}[1]{ \begin{myBox}[adjusted title=#1, sharp corners]}{\end{myBox}}



% ROTATION COMMAND \rotraise{}
\newcounter{idx}
\newcounter{posx}
\newcommand{\myqed}{\strut\hfill$\qed$\\}
\newcommand{\rotraise}[1]{%
	\StrLen{#1}[\slen]
	\forloop[-1]{idx}{\slen}{\value{idx}>0}{%
		\StrChar{#1}{\value{idx}}[\crtLetter]%
		\IfSubStr{tlQWERTZUIOPLKJHGFDSAYXCVBNM}{\crtLetter}
		{\raisebox{\depth}{\rotatebox{180}{\crtLetter}}}
		{\raisebox{1ex}{\rotatebox{180}{\crtLetter}}}}%
}





%% Style Definitions %%%%%%%%%%%%%%%%%%%%%%%%%%%%%%%%%%%%%%%%%%%%%%%%%%%%%%%%%%
\numberwithin{equation}{section}

% My Theorem Style ============================================================

\newtheoremstyle{myThmStyle}% name of the theorem
{6pt}% habovespacei
{6pt}% hbelowspacei
{\itshape}% font
{}% hindenti
{}% hheadfonti
{}% hheadpuncti
{ }% hheadspacei
{\thmname{\normalfont\large\textsc{#1 \thmnumber{#2}} \normalsize} - \thmnote{#3:}\\ %\\[1pt]\thmnumber{ \hfill \normalfont(#2)}\\[1pt]
}

% Frame Styles ================================================================

\definecolor{thmBgColor}{RGB}{250,250,250}
\definecolor{thmLnColor}{RGB}{200,200,200}

\mdfdefinestyle{MDFStyGrayScreen}{%
	linecolor=thmLnColor,
	backgroundcolor=thmBgColor,
	linewidth=1pt,
	topline=true,
	bottomline=true,
	rightline=false,
	leftline=false,
	outerlinewidth=2pt,
	roundcorner=0pt,
	innertopmargin=4pt, %\baselineskip
	innerbottommargin=4pt, %\baselineskip,
	innerrightmargin=3pt,
	innerleftmargin=3pt,
	skipabove=\topskip,
	skipbelow=\topskip,
	nobreak=true
}

% Thorems =====================================================================

\theoremstyle{myThmStyle}
\newtheorem{myTheorem}{theorem}%[section]
%\newtheorem{myTheorem}[equation]{theorem}

% Equations environment ===========================================================
\renewcommand\theequation{Eq.~\arabic{equation}}
\newcommand{\taglabel}[1]{\tag{#1}\label{#1}}
\newcommand{\taglabels}[2]{\tag{#1}\label{#2}}

% Proof environment ===========================================================
\renewenvironment{proof}{\paragraph{Proof:}}{\myqed}


\newenvironment{myTheoremEnvEx}[1][\hsize]
{% 
	\def\FrameCommand 
	{%
		{\color{thmLnColor}\vrule width 8pt}%
		\hspace{0pt}%must no space.
		\fboxsep=5pt\colorbox{white}%
	}%
	\vspace{-\topskip}
	\vspace{-\topskip}
	\vspace{-1pt}
	\MakeFramed{\hsize#1\advance\hsize-\width\noindent\FrameRestore}%
}
{\endMakeFramed}
\setlength{\FrameSep}{0pt}

% Theorem environment =========================================================

\newenvironment{theorem}[1]
{
	\begin{mdframed}[style=MDFStyGrayScreen]
		\begin{myTheorem}#1
		}
		{
		\end{myTheorem}
	\end{mdframed}
}

\newenvironment{changemargin}[2]{%
	\begin{list}{}{%
			\setlength{\topsep}{0pt}%
			\setlength{\topmargin}{#1}%
			\setlength{\bottommargin}{#2}%
%			\setlength{\leftmargin}{#3}%
%			\setlength{\rightmargin}{#4}%
%			\setlength{\listparindent}{\parindent}%
%			\setlength{\itemindent}{\parindent}%
%			\setlength{\parsep}{\parskip}%
		}%
		\item[]}{\end{list}
	}

\graphicspath{{images/}} % Lege den Standardpfad mit Bilder fest
\DeclareGraphicsExtensions{.pdf,.png,.jpg} % bevorzuge pdf-Dateien vor den anderen